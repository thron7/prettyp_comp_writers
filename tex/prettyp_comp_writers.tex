\documentclass[11pt,a4paper]{article}

\begin{document}
\title{Pretty-Printing for Compiler Writers}
\author{
Thomas Herchenroeder\\
1\&1 Internet AG
}

\date{\today}
\maketitle

\begin{abstract} 
Pretty-printing, i.e. the regularly formatting of source code
in terms of line breaks and indentation, poses extra challenges to compiler
writers, namely handling of code comments. This paper discusses problems and
approaches to address these. Particularly, it proposes to use a \emph{Concrete
Syntax Tree} to capture and maintain code comments, which are usually not
included in parse results. But they are needed to restore the text of parsed
source code faithfully.
\end{abstract}

\section{The Problem}
Pretty-printing is a little discussed issue, although it is not
far-fetched. Many IDEs provide automatic code layout, as well do many command
line tools. But it poses a significant
challenge: the handling of code comments. Parser and compiler writers
frequently waive their hands, as comments in code are treated as white space and
are usually ignored during parsing.

\section{Approaches}

\section{The best Approximation: Concrete Syntax Trees}

\bibliography{}

\end{document}
